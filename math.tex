\documentclass{article}
\usepackage{amsmath} % for mathematical symbols and environments
\usepackage{amssymb} % for additional mathematical symbols like \mathbb

\title{Stochastic Processes and Option Pricing}
\author{Dylan Tirandaz}
\date{\today}

\begin{document}

\maketitle

\section{Introduction}

In this document, we explore the application of stochastic processes, specifically Geometric Brownian Motion (GBM), in option pricing.

\section{Theory of Geometric Brownian Motion}

Geometric Brownian Motion (GBM) is defined by the stochastic differential equation:
\begin{equation}
dS_t = \mu S_t dt + \sigma S_t dW_t
\end{equation}
where:
\begin{itemize}
    \item \( S_t \) is the asset price at time \( t \),
    \item \( \mu \) is the drift rate (expected return of the asset),
    \item \( \sigma \) is the volatility (standard deviation of the asset's returns),
    \item \( W_t \) is a Wiener process (standard Brownian motion).
\end{itemize}

The solution to this SDE is:
\begin{equation}
S_t = S_0 \exp \left( \left( \mu - \frac{\sigma^2}{2} \right)t + \sigma W_t \right)
\end{equation}

\section{Option Pricing using GBM}

To price European options under GBM, we simulate the asset price using Monte Carlo methods. The price of a European Call option can be estimated as:
\begin{equation}
C(S_0, K, r, \sigma, T) = e^{-rT} \mathbb{E} \left[ \max(S_T - K, 0) \right]
\end{equation}
where \( S_0 \) is the initial stock price, \( K \) is the strike price, \( r \) is the risk-free interest rate, \( \sigma \) is the volatility, \( T \) is the time to expiration, and \( S_T \) is the asset price at expiration.

\section{Implementation in C++}

The following C++ code implements the simulation of GBM and calculates the price of a European Call option:

\begin{verbatim}
// Function to simulate Geometric Brownian Motion (GBM)
double geometric_brownian_motion(double S0, double drift, double volatility, double T, double dt) {
    double S = S0;
    for (double t = 0; t < T; t += dt) {
        double W = random_normal(0.0, 1.0);
        S *= exp((drift - 0.5 * volatility * volatility) * dt + volatility * sqrt(dt) * W);
    }
    return S;
}

// Function to calculate European Call option price using GBM
double european_call_option_price(double S0, double K, double r, double volatility, double T) {
    // Parameters
    int num_simulations = 100000; // Number of Monte Carlo simulations
    double dt = T; // Time steps (assuming annualized)
    double discounted_expected_payoff = 0.0;

    // Simulate asset prices and calculate option payoffs
    for (int i = 0; i < num_simulations; ++i) {
        double S_T = geometric_brownian_motion(S0, r, volatility, T, dt);
        double payoff = max(S_T - K, 0.0); // Payoff for European Call option
        discounted_expected_payoff += payoff / num_simulations;
    }

    // Discounted expected payoff to present value
    double option_price = exp(-r * T) * discounted_expected_payoff;

    return option_price;
}\end{verbatim}

\section{Results}

The C++ program produces the following results for the European Call option:

\begin{itemize}
    \item Initial stock price \( S_0 = 100 \)
    \item Strike price \( K = 100 \)
    \item Risk-free interest rate \( r = 0.05 \)
    \item Volatility \( \sigma = 0.2 \)
    \item Time to expiration \( T = 1 \) year
\end{itemize}

The European Call option price is \( C(S_0, K, r, \sigma, T) = \) [insert calculated option price here].

\section{Conclusion}

This document demonstrates the use of Geometric Brownian Motion in simulating asset prices and pricing European options.

\end{document}
